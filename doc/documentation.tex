\documentclass[a4paper]{report}

\usepackage[top=25mm,bottom=25mm]{geometry}
\usepackage[utf8]{inputenc}
\usepackage[T1]{fontenc}

\usepackage{amssymb}
%\usepackage{graphicx}

\usepackage{tikz}
\usetikzlibrary{automata, positioning, arrows}

\tikzset{
  ->,  % makes the edges directed
  %>=stealth’, % makes the arrow heads bold
  node distance=3cm, % specifies the minimum distance between two nodes. Change if necessary.
  every state/.style={thick, fill=gray!10}, % sets the properties for each ’state’ node
  initial text=$ $, % sets the text that appears on the start arrow
}

\title{Handcart for Fenice}
\author{Matteo Bitussi - Lorenzo Rossi\\ E-Agle TRT}
\date{2020-2021}

\begin{document}
  \maketitle

  \tableofcontents

  \section*{Introduction}
  Cose

  \chapter{General view}

  \chapter{Back-end}
  The back-end is thought to act as the controller to start and stop the charge, to handle the fans and to act as an intermedery between the BMS and the BRUSA.
  \section{How it works}
  \begin{itemize}
    \item The back-end sends a can message to the BMS, with \textbf{Cut-off voltage}, \textbf{type of charge} (fast or normal)
    \item The accumulator decides what charging curve to follow, he sends charging messages to the back-end that forward them to the BRUSA
    \item The charge can be interupped by the back-end itself, or finished/interrupped by the BMS
  \end{itemize}
  Note that the accumulator has a parallel state-machine (apart the normal one) when is charging

  \section{Settings}
  There are various settings to be choosen in the back-end
  \begin{itemize}
    \item \textbf{Charging speed}: back-end can ask the BMS to use a particular charging curve (\textbf{fast} or \textbf{normal}) default is normal.
    \item \textbf{Current drawn from the outlet}: back-end can ask BRUSA to set a maximum current to drawn from the outlet, useful when using standard home outlet
    \item \textbf{Fan profile}: back-end can use a fixed profile for the fans (i.e 90), by default it uses a fan curve
    \item \textbf{Choose Accumulator}: this setting is obligatory, the back-end has to know what car's accumulator is attached
  \end{itemize}

  \section{The state machine}
  The main.py is based on a state machine, which states are these

    \begin{figure}[ht] % ’ht’ tells LaTeX to place the figure ’here’ or at the top of the page
      \centering % centers the figure
      \begin{tikzpicture}% tikz code goes here
        \node[state, initial] (check) {$check$};
        \node[state, right of=check] (idle) {$idle$};
        \node[state, right of=idle] (precharge) {$precharge$};
        T\node[state, right of=precharge] (ready) {$ready$};
        \node[state, right of=ready] (charge) {$charge$};
        \node[state, below of=charge] (c_done) {$c\_done$};
        \node[state, below of=check] (error) {$error$};
        \node[state, below of=precharge] (end) {$end$};

        \draw
          (check) edge[loop above] (check)
          (check) edge[above] (idle)
          (idle) edge[loop above] (idle)
          (idle) edge[above] (precharge)
          (precharge) edge[above] (ready)
          (ready) edge[above] (charge)
          (ready) edge[loop above] (ready)
          (charge) edge[bend left, below] (c_done)
          (charge) edge[loop above] (charge)
          (c_done) edge[loop right] (c_done)
          (error) edge[below] (check)
          (error) edge[below] (end)
          (error) edge[loop left] (error)
          (c_done) edge[left] (ready)
          (c_done) edge[bend left, above] (charge)
      \end{tikzpicture}
      \caption{Back-end state machine}
      \label{fig:my_label}
    \end{figure}

  \subsection{STATE: check}
  Checks for the presence of the accumulator and BRUSA
  \subsection{STATE: idle}
  Accumulator and BRUSA are connected, wait for user input to start precharge
  \subsection{STATE: precharge}
  Ask BMS TS\_ON
  \subsection{STATE: ready}
  Accumulator has closed airs, and is ready to be charged. Waiting for user input
  \subsection{STATE: charge}
  Charging, until bms stop or user stop
  \subsection{STATE: c\_done}
  Looping waiting for user input
  \subsection{STATE: error}
  An error has ben detected, tell to user and restart program or exit

  \chapter{Front-end}
  \section{General description}
  The front-end GUI is based on Qt, written in python. It has the purpose to act as a interface for the user to manage the charge process

  \chapter{Deamon gRPC server}
  \section{General description}
  The server is needed from the back-end and the front-end to communicate. It is based on gRPC and protocol Buffers

\end{document}
